\gdef\problemNumber{The Josephus Problem}
\def\problemText{
Suppose we have $n$ items arranged in a circle as shown:\\
\begin{figure}[H]
  \caption{Numbered items arranged in a circle}
\begin{tikzpicture}
  \draw circle (1cm);
  \draw (170:1.5) node {$n-1$};
  \draw (135:1.5) node {$n$};
  \draw (90:1.5) node {$1$};
  \draw (45:1.5) node {$2$};
  \draw (10:1.5) node {$3$};
  \draw (270:1.5) node {$\cdots$};
\end{tikzpicture}
\end{figure}
\,\\
We proceed around the circle, removing every other item (item $2$ is the first to be
removed) until one item remains. For example, if $n=10$, the sequence of removed
items is\\$2,4,6,8,10,3,7,1,9$, with item $5$ being the last one left. The
\emph{survivor's number}, $J(n)$, is the number of the last remaining item from a
set of $n$ items (thus, $J(10)=5$).
\begin{list}{\textbf{\alph{enumii}.}}{\usecounter{enumii}}
  \item Suppose there are $2n$ items. After $n$ items have been removed, there
  are $n$ items remaining. What are the numbers of the items that remain? How do
  those numbers relate to the numbering used for an initial set of $n$? Use this
  information to express $J(2n)$ in terms of $J(n)$.
  \item Suppose there are now $2n+1$ items. After $n+1$ items have been removed, there
  are $n$ items remaining. What are the numbers of the items that remain? How do
  those numbers relate to the numbering used for an initial set of $n$? Use this
  information to express $J(2n+1)$ in terms of $J(n)$.
  \item Combine parts {\bf a} and {\bf b}, along with the base case $J(1)=1$, to
  form a recurrence relation for $J(n)$.
  \item Use the recurrence relation to create a small table of $J(n)$ values (you
  shouldn't need more than 20 to see the pattern). Use this table to find a closed
  form (\emph{i.e.}, non-recursive) for $J(n)$.
  \emph{Hint}: Express $n$ as $n=2^m+l$, where $2^m\le n<2^{m+1}$.
  \item Prove that your closed form solution is correct. \emph{Hint}: Use induction
  on $m$.
\end{list}
}
\def\problemSolution {
\textcolor{blue}{
\textbf{Answer:}\\[6pt]
a. We are left with odd numners. J(2n) = 2J(n) - 1 \\[6pt]
b. We are left with even numbers.  J(2n+1) = 2J(n) + 1 \\[6pt]
c. Combining a and b: \\[6pt]
\[
J(n) = \begin{cases}
J(1) = 1 & \text{base case } \\[6pt]
2J(\frac{n}{2}) - 1 & \text{if } n \text{ is even} \\[6pt]
2J(\lfloor \frac{n}{2} \rfloor) + 1 & \text{if } n \text{ is odd and $n > 1$}
\end{cases}
\]\\[6pt]
d. We found from the table that if n is power of 2 then our J(n) is always 1. \\[6pt]
We will use hint for other number. \\\emph{Hint}: Express $n$ as $n=2^m+l$, where $2^m\le n<2^{m+1}$.\\[6pt]
$l = n - 2^m$ \\[6pt]
Our closed form: $J(n) = 2l + 1$ where $n = 2^m + l$ and $0 \le l < 2^m$\\[12pt]
e. To show closed form solution is correct: \\[6pt]
Base Case: m = 0\\
Then, $2^0 = 1$\\
So l = 0\\
Inductive Hypothesis:\\
}
}
\gdef\problemOutcomes{CS 6, 5870-1, 5870-2}
