\gdef\problemNumber{GT C - 11.3}
\def\problemText{
There is a sorting algorithm, ``Stooge-sort,'' which is named after the comedy
team, ``The Three Stooges.'' If the input size, $n$, is $1$ or $2$, then the
algorithm sorts the input immediately. Otherwise, it recursively sorts the first
$2n/3$ elements, then the last $2n/3$ elements, and then the first $2n/3$
elements again. The details are shown in the following algorithm. Show that
Stooge-sort is correct and characterize the running time, $T(n)$, for
Stooge-sort, using a recurrence equation, and use the master theorem to
determine an asymptotic bound for $T(n)$.\\[12pt]
\begin{algorithm}[H]
    \caption{Stooge Sort}
    \begin{algorithmic}[1]
        \Procedure{StoogeSort}{${\bf Comparable\ }A[],{\bf int\ }low,
            {\bf int\ }high$}
            \State $n\gets high-low+1$
            \Comment\textcolor{blue}{Determine size of list}
            \Statex
            \If{$n=2$}
                \If{$A[low]>A[high]$}
                    \State {\sc Swap}$(A[low],A[high])$
                \EndIf
            \ElsIf{$n>2$}
                \State $m\gets\lfloor n/3\rfloor$
                \Comment\textcolor{blue}{Calculate overlap}
                \Statex
                \State {\sc StoogeSort}$(A,low,high-m)$
                \Comment\textcolor{blue}{Recursively sort first part}
                \State {\sc StoogeSort}$(A,low+m,high)$
                \Comment\textcolor{blue}{Recursively sort second part}
                \State {\sc StoogeSort}$(A,low,high-m)$
                \Comment\textcolor{blue}{Recursively sort first part again}
            \EndIf
        \EndProcedure
    \end{algorithmic}
\end{algorithm}
}
\def\problemSolution{
\textcolor{blue}{
\textbf{Answer:}\\[6pt]
a. Show that Stooge-sort is correct \\[6pt]
If $n < 2$, then do nothing. That is , if there is only one item or no item the list/array is sorted.\\[6pt]
Base Case: $n = 2$: \\ it compares and swap elements if needed. \\[6pt]
Thefore, the subarray is sorted. \\[6pt]
Inductive Hypothesis: \\Assume that for all $k < n$, this algorithm holds true.  \\[6pt]
For $n > 2:$  \\[6pt]
Let's divide the array into 3 sections: A, B and C. \\[6pt]
i. First Recursive Call: StoogeSort(A[1...m])\\
A and B are sorted [Structural Induction]\\
We can say that everything in B is larger than in A. \\[6pt]
ii. Second Recursive Call: StoogeSort(A[n-m+1...n])\\
This makes sure that largest elements are moved to C section. \\
Everythign in C is larger than everything in A and B \\
There could have been small elements in part C that needed to over A and B and not there yet.\\[6pt]
iii. Third Recursive Call: StoogeSort(A[1...m])\\
This sort the elements in A and B again. The A and B contains smallest m elements, sorting this puts elements in correct position \\
Since the second recursive call makes sure that C has the largest elements and are sorted correctly.\\
Then final recursive call makes sure that the elements in A and B are sorted. We can say that Stooge-sort is correct. \\
\\[6pt]b. Determine an asymptotic bount for T(n)\\[6pt]
$T(n) = 3T(\frac{2}{3}n) + 0$ \\[6pt]
So, $a = 3, b = \frac{3}{2}, F(n) = 0, lg_ba \approx 2.71$ \\[6pt]
As per M.T Case 1, $F(n) = 0$ is smaller than $O(n^{2.71})$  \\[6pt]
Therefore, $T(n) = O(n^{2.71})$ \\[6pt]
}
}
