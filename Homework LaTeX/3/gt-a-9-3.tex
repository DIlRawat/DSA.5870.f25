\gdef\problemNumber{GT A-9.3}
\def\problemText{
Suppose you are the postmaster in charge of putting a new post office in a small
town, where all the houses are along one street, where the new post office
should go as well. Let us view this street as a line and the houses on it as a
set of real numbers, $\lbrace x_1,x_2,\ldots,x_n\rbrace$, corresponding to
points on this line. To make everyone in town as happy as possible, the
location, $p$, for the new post office should minimize the sum
\[\sum_{i=1}^n|p-x_i|\]
Describe an efficient algorithm for finding the optimal location for the new
post office, show that your algorithm is correct, and analyze its running time.
\\[12pt]
}
\def\problemSolution{
\textcolor{blue}{
\textbf{Answer:}\\[6pt]
I am using Quick Select Algorithm to find the median and place the lamp post there.\\[6pt]
\begin{algorithm}
    \caption{quickSelect(S, m)}
    \begin{algorithmic}[1]
        \Require Input: S as Set of n real numbers that corresponds a house on a line(street) , and an integer $m = \lceil n/2 \rceil$.
        \Ensure Output: Location p for the new post office that minimize the sum distance from each house to the post office p. 
        \If {$n = 1$}
            \State return first element of S as the location for a new post office.
        \EndIf
        \State pick a random element x of S
        \State remove all the emelents from S and put them into 3 sequences:
        \begin{itemize}
            \item L, storing the elements in S less than x.
            \item E, storing the elements in S sequal to x.
            \item G, storing the elements in S greater than x.
        \end{itemize}
        \If {$m \le |L|$}
            \State quickSelect({$L, m$})
        \ElsIf {$m \le |L| + |E|$}
            \State return $x$
        \Else
            \State quickSelect({$G, m - |L| - |E|$})
        \EndIf
    \end{algorithmic}
\end{algorithm}
\\[6pt] We place the post across the street when there is only one house. If two houses we can put the post
in between houses. The sum of the distance to post office from those two houses is minimum if we put
the post office in between those two houses. \\[6pt]
If we got more than two houses, we find the median and keep going until we are left with one or two houses\\[6pt]
\\[6pt] Analysis: It depends on selected pivot. We are selecting it randomly in the above algorithm\\[6pt]
Worst Case: If the pivot is the smallest or largest item in the Sequence S with n elements.\\[6pt]
This means either: \\[6pt]
$L = 0$, $|G| = n - 1$ or $L = n - 1$, $|G| = 0$ \\[6pt]
Since this involves recusion, we will need a recurrence equation:\\[6pt]
$T(n)$ = Time to partition (L, E, R) + T(Size of Subproblem)\\[6pt]
$ T(n) = O(n) + T(p)$, where p is the partion \\[6pt]
$ T(n) = T(n - 1) + O(n)$
We can use subsutition to solve this recurrence equation which will give us: \\[6pt]
$T(n) = 1 + 2 + 3 + \dots + n = \frac{n(n+1)}{2}$ \\[6pt]
This equates to: $T(n) = O(n^2)$
\\[12pt]
Average Case: In this case the pivot partition the sequence S at most 3n/4 \\[6pt]
$T(n) = T(\frac{3n}{4}) + O(n)$ \\[6pt]
We can use master theorem to solve this: \\[6pt]
$a = 1, b = 4/3, log_ba = 1, f(n) = O(n)$ \\[6pt]
As per Case 3: $O(n)$ \\[6pt]
Therefore on average, the quick select algorithm runs on linear time. 
}
\\[6pt]
}
\gdef\problemOutcomes{CS 6, 5870-1}
