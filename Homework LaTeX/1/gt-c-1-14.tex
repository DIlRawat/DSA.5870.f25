\gdef\problemNumber{GT C-1.14}
\def\problemText{
A $n$-degree polynomial $p(x)$ is an equation of the form
\[p(x)=\sum_{i=0}^na_ix^i\]
where $x$ is a real number and each $a_i$ is a constant.
\begin{list}{\textbf{\alph{enumii}}}{\usecounter{enumii}}
    \item Describe a simple $O(n^2)$-time method for computing $p(x)$ for a
    particular value of $x$.
    \item Consider now a rewriting of $p(x)$ as
    \[p(x)=a_0+x\left(a_1+x\left(a_2+x\left(a_3+\cdots+x\left(a_{n-1}+xa_n
    \right)\cdots\right)\right)\right)\]
    which is known as \emph{Horner's method}. Using the $O(\,)$ notation,
    characterize the number of multiplications and additions this method of
    evaluation uses.\\[12pt]
\end{list}
\vskip12pt
}

\def\problemSolution{*
{
\color{blue}
\textbf{Answer:}\\[6pt]
\begin{list}{\textbf{\alph{myenumi})}}{\usecounter{myenumi}\setcounter{myenumi}{0}}
\item $O(n^2)$ Method:
A brute force method is to compute each term $a_ix^i$ individually and sum them up. 
The term $a_ix^i$ can be computed by multiplying $x$ by itself $i$ times. 
Here we have nested loop structure.
\par p(x) = $a_0x^0$ + $a_1x^1$ + ... + $a_nx^n$
\begin{algorithm}[H]
    \caption{Simple $O(n^2)$ Polynomial Evaluation}
    \begin{algorithmic}
         \color{blue}
        \State $result \gets 0$
        \For{$i \gets 0$ to $n$}
            \State $termPower \gets 1$
            \For{$j \gets 1$ to $i$}
                \State $termPower \gets termPower \cdot x$
            \EndFor
            \State $termTotal \gets a_i \cdot termPower$
            \State $result \gets result + termTotal$
        \EndFor
    \State \Return $result$
    \end{algorithmic}
\end{algorithm}

\item Horner's Method ($O(n)$):
Horner's method is a more efficient way to evaluate the polynomial. 
It evaluates the nested form of the equation from the inside out.
\textbf{(b) Horner’s method ($O(n)$):}
\begin{algorithm}[H]
    \caption{Horner's Method}
    \begin{algorithmic}
         \color{blue}
        \State $result \gets a_n$
        \For{$i \gets n-1$ downto $0$}
            \State $result \gets a_i + x \cdot result$
        \EndFor
        \State \Return $result$
    \end{algorithmic}
\end{algorithm}

We start with $result \gets a_n$ We iterate backwards, $i = n - 1$ to $0$:
\par In every single iteration, we do: \newline
    1 multiplication:  $x \cdot result$ \newline
    1 addition:  $a_i + ...$

\par That means we do $n$ multiplication and $n$ addition for $n$ degree polynomial. 
\par Total work = n + n = 2n which we can say : O(n)

 \end{list}
}
}
\gdef\problemOutcomes{CS 2, CS 6, 5870-1, 5870-2}