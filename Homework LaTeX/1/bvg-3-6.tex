\gdef\problemNumber{BvG 3.6 (modified)}
\def\problemText{
In this exercise, all integers are considered to be nonnegative, for simplicity.
A \emph{divisor} of an integer $k$ is any integer $d\ne0$ such that $k/d$ has no
remainder. A \emph{common divisor} for a set of integers is an integer that is a
divisor for each integer in the set. Euclid's algorithm for finding the greatest
common divisor (GCD) of two nonnegative integers, $m$ and $n$, can be written as
follows:
\begin{algorithm}[H]
\begin{algorithmic}[1]
\Procedure{gcd}{{\bf int }$m$,{\bf int }$n$}
\If{$n=0$}
\State $answer\gets m$
\ElsIf{$m<n$}
\State $answer\gets\textsc{gcd}(n,m)$
\Else
\State $r\gets m-n\cdot\lfloor \frac{m}{n}\rfloor$
\Comment $r$ is the remainder of $\frac{m}{n}$
\State $answer\gets\textsc{gcd}(n,r)$
\EndIf
\State {\bf return }$answer$
\EndProcedure
\end{algorithmic}
\end{algorithm}
The preconditions for {\sc gcd}$(m,n)$ are that $m\ge0,n\ge0$ and $m+n>0$.
Prove the following using induction.
\begin{list}{\textbf{\alph{enumii}.}}{\usecounter{enumii}}
	\item If the preconditions of {\sc gcd}$(m,n)$ are satisfied, then the value
	that the function returns is \emph{some} common divisor of $m$ and $n$.
	\item If the preconditions of {\sc gcd}$(m,n)$ are satisfied, then the value
	that the function returns is the \emph{greatest} common divisor of $m$ and $n$.
\end{list}
\emph{Hints}\/: If $d$ is a divisor of $k$, how can you rewrite $k$ in
terms of $d$\/? How do you show that two sets are equal?\\[12pt]
}
\def\problemSolution{
\textcolor{blue}{
\textbf{Answer:}\\[6pt]
Given preconditions: $m \ge 0, n \ge 0 $ and $m + n > 0 $
\begin{list}{\textbf{\alph{enumii}.}}{\usecounter{enumii}}
\item Base case: $m > 0 $ and $n = 0$ . $GCD(m,0) \gets m : line 2$. We can say that m is the common divisor since we can divide m and 0 by m. 
	If (m,n)  = GCD(1,0), then it returns m = 1 which divides both 0 and 1
\item Base case: $m = 0 $ and $n > 0$ . $GCD(0,n) \gets GCD(n,0)$ since $m < n$ : line 4. 
	We can say that n is the common divisor since we can divide 0 and n by n. If (m,n)  = GCD(0,1), 
	since $m < n$ so we call GCD(1,0) then it returns 1 which divides both 0 and 1
\item Therefore, the returned value is common divisor in above cases.
\end{list}
Let s = m + n for a pair (m,n).
Now assume the claim holds for every pair (x,y) with $x + y < s$.
\begin{list}{\textbf{\alph{enumii}.}}{\usecounter{enumii}}
	\item Case: $m < n$ calls GCD(n,m) which reduces the 
	\item Case $m \ge n > 0$. We call GCD(n,r), where $r = m - n . \lfloor \frac{m}{n} \rfloor$. 
		The remainder r is: $  0 \le r < n$. Then $n + r \le m + n = s$. 
		Let's say value returned from GCD(n,r) = d which divides both n and r. This means $d\mid n$ and $d\mid r$
\end{list}
}
}
\gdef\problemOutcomes{CS 1, CS 6, 5870-1}

